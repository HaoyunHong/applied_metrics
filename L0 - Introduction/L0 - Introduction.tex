\documentclass[10pt,aspectratio=169]{beamer}

% silence some Metropolis warnings
\usepackage{silence}
\WarningFilter{beamerthememetropolis}{You need to compile with XeLaTeX or LuaLaTeX}
\WarningFilter{latexfont}{Font shape}
\WarningFilter{latexfont}{Some font}

% define custom colors
\usepackage{xcolor}
\definecolor{dark gray}{HTML}{444444}
\definecolor{light gray}{HTML}{777777}
\definecolor{dark red}{HTML}{BB0000}
\definecolor{dark green}{HTML}{00BB00}
\definecolor{RoyalBlue}{cmyk}{1, 0.50, 0, 0}

% configure metropolis
\usetheme[numbering=fraction]{metropolis}
\setbeamercolor{background canvas}{bg=white}
\setbeamercolor{frametitle}{bg=dark gray}
\setbeamercolor{alerted text}{fg=dark red}
\setbeamercolor{item projected}{bg=dark red}
\setbeamercolor{local structure}{fg=dark red}
\setbeamersize{text margin left=0.5cm,text margin right=0.5cm}
\setbeamercovered{transparent=10}

% use thicker lines
\makeatletter
\setlength{\metropolis@titleseparator@linewidth}{1pt}
\setlength{\metropolis@progressonsectionpage@linewidth}{1pt}
\makeatother

% custom bullet points
\setbeamertemplate{itemize item}{\color{dark red}$\blacktriangleright$}
\setbeamertemplate{itemize subitem}{\color{dark red}$\blacktriangleright$}
\setbeamertemplate{itemize subsubitem}{\color{dark red}$\blacktriangleright$}
\newcommand{\custombullet}{{\color{dark red}$\blacktriangleright$}\hspace{0.5em}}


% imports
\usepackage[english]{babel}
\usepackage[utf8]{inputenc}
\usepackage{amsthm}
\usepackage{amssymb}
\usepackage{amsmath}
\usepackage{amsfonts}
\usepackage{mathtools}
\usepackage{mathabx}
\usepackage{stmaryrd}
\usepackage{graphicx}
\usepackage{hyperref}
\usepackage{xfrac}
\usepackage{appendixnumberbeamer}
\usepackage{tabularx}
\usepackage{listings}

% for code formatting
   \lstset{language=R,
           basicstyle=\ttfamily\scriptsize,
           keywordstyle=\color{blue}\ttfamily,
           stringstyle=\color{red}\ttfamily,
           commentstyle=\color{green}\ttfamily,
          breaklines=true
          }



% check and x marks
\usepackage{pifont}
\newcommand{\cmark}{{\color{dark green}\ding{51}}\hspace{0.3em}}
\newcommand{\xmark}{{\color{dark red}\ding{55}}\hspace{0.5em}}


% use classic font for math
\usepackage[T1]{fontenc} % Needed for Type1 Concrete \usepackage{concmath}
\usefonttheme{serif}
\usefonttheme{professionalfonts}
\usepackage{concmath}
\setbeamerfont{equation}{size=\tiny}



% diagrams
\usepackage{tikz}
\usetikzlibrary{decorations.pathreplacing, arrows, shapes, patterns, angles, quotes}

% references
\usepackage[natbibapa]{apacite}
\bibliographystyle{apacite}
\renewcommand{\bibsection}{}

% use ampersands instead of "and" for text citations
\AtBeginDocument{\renewcommand{\BBAB}{\&}}

% possessive cites
\makeatletter
\patchcmd{\NAT@test}{\else \NAT@nm}{\else \NAT@nmfmt{\NAT@nm}}{}{}
\DeclareRobustCommand\citepos
  {\begingroup
   \let\NAT@nmfmt\NAT@posfmt
   \NAT@swafalse\let\NAT@ctype\z@\NAT@partrue
   \@ifstar{\NAT@fulltrue\NAT@citetp}{\NAT@fullfalse\NAT@citetp}}
\let\NAT@orig@nmfmt\NAT@nmfmt
\def\NAT@posfmt#1{\NAT@orig@nmfmt{#1's}}
\makeatother

% spaced-out lists
\newenvironment{wideitemize}{\itemize\addtolength{\itemsep}{10pt}}{\enditemize}
\newenvironment{wideenumerate}{\enumerate\addtolength{\itemsep}{10pt}}{\endenumerate}

% replace footnotes with buttons
\usepackage[absolute,overlay]{textpos}
\newcounter{beamerpausessave}
\newcommand{\always}[1]{
    \setcounter{beamerpausessave}{\value{beamerpauses}}
    \setcounter{beamerpauses}{0}
    \pause
    #1 
    \setcounter{beamerpauses}{\value{beamerpausessave}}
    \addtocounter{beamerpauses}{-1}
    \pause
}
\newcommand{\buttons}[1]{\always{
    \begin{textblock*}{\paperwidth}(0.015\textwidth, 1.022\textheight)
        \scriptsize
        #1
    \end{textblock*}
}}
\newcommand{\appendixbuttons}[1]{\always{
    \begin{textblock*}{\paperwidth}(0.015\textwidth, 1.043\textheight)
        \scriptsize
        #1
    \end{textblock*}
}}
\newcommand{\goto}[2]{\hyperlink{#1}{{\color{dark red}$\smalltriangleright$} #2}\hspace{0.5em}}
\newcommand{\goback}[2]{\hyperlink{#1}{{\color{dark red}$\smalltriangleleft$} #2}\hspace{0.5em}}

% custom appendix
\renewcommand{\appendixname}{\texorpdfstring{\translate{Appendix}}{Appendix}}

% change color of cites and URLs
\let\oldcite\cite
\let\oldcitet\citet
\let\oldcitep\citep
\let\oldcitepos\citepos
\let\oldcitetalias\citetalias
\let\oldcitepalias\citepalias
\let\oldurl\url
\def\cite#1#{\citeaux{#1}}
\def\citet#1#{\citetaux{#1}}
\def\citep#1#{\citepaux{#1}}
\def\citepos#1#{\citeposaux{#1}}
\def\citetalias#1#{\citetaliasaux{#1}}
\def\citepalias#1#{\citepaliasaux{#1}}
\def\url#1#{\urlaux{#1}}
\newcommand*\citeaux[2]{{\color{light gray}\oldcite#1{#2}}}
\newcommand*\citetaux[2]{{\color{light gray}\oldcitet#1{#2}}}
\newcommand*\citepaux[2]{{\color{light gray}\oldcitep#1{#2}}}
\newcommand*\urlaux[2]{{\color{light gray}\oldurl#1{#2}}}
\newcommand*\citeposaux[2]{{\color{light gray}\oldcitepos#1{#2}}}
\newcommand*\citetaliasaux[2]{{\color{light gray}\oldcitetalias#1{#2}}}
\newcommand*\citepaliasaux[2]{{\color{light gray}\oldcitepalias#1{#2}}}

% custom math commands
\DeclareMathOperator*{\argmax}{argmax}
\DeclareMathOperator*{\argmin}{argmin}
\renewcommand{\Pr}{\mathbb{P}}
\newcommand{\E}{\mathbb{E}}
\newcommand{\Var}{\mathbb{V}}
\newcommand{\Cov}{\mathbb{C}}
\newcommand{\overbar}[1]{\mkern 1.5mu\overline{\mkern-1.5mu#1\mkern-1.5mu}\mkern 1.5mu}
\newcommand{\abs}[1]{\lvert#1\rvert}
\newcommand{\norm}[1]{\lVert#1\rVert}

% tables
\usepackage{booktabs}
\usepackage{colortbl}
\usepackage{multirow}
\usepackage{makecell}
\arrayrulecolor{dark red}

% custom date
\usepackage{datetime}
\newdateformat{monthyeardate}{\monthname[\THEMONTH] \THEYEAR}

% fix pauses with graphics
\usepackage{../resources/fixpauseincludegraphics}


% \documentclass[english,xcolor={dvipsnames},aspectratio=169]{beamer}

% %\usepackage[T1]{fontenc}
% %\usepackage{libertine}
% \usepackage[italic]{mathastext}

% \usepackage{float}
% \usepackage{amsmath}
% \usepackage{amssymb}
% \usepackage{graphicx}
% \usepackage{setspace}
% \usepackage[outdir=./]{epstopdf}

% \usepackage{caption,subcaption}
% \usepackage{booktabs}
% \usepackage[flushleft]{threeparttable}

% \usepackage{siunitx}
% \sisetup{
% input-symbols = {()},
% group-digits  = false,
% explicit-sign
% }

% %\usetheme{CambridgeUS}
% \usetheme{Boadilla}
% \usecolortheme{seahorse}

% \setbeamertemplate{itemize subsubitem}{\tiny\raise1.5pt\hbox{\donotcoloroutermaths$\blacktriangleright$}}


% \usepackage{tikz}
% \usetikzlibrary{arrows, shapes, snakes, patterns, angles, quotes}
% \usetikzlibrary{tikzmark}

% \makeatletter


% \usepackage{setspace}
% \usepackage[normalem]{ulem}
% \setbeamersize{text margin left=5pt,text margin right=5pt}

% \makeatother

% \usepackage{babel}
\begin{document}

\title[L0 - Introduction]{ Econometrics I}
\subtitle{Lecture 0: Introduction}
\author{Chris Conlon \\NYU Stern}
\institute{Grad IO }
\date{Fall 2025}
\maketitle

\section{Econometrics Overview}


\begin{frame}{Course Description}

\parbox{\linewidth}{
The aim of the course is to teach you to {\bf use} popular
applied econometric methods while developing your theoretical understanding
of those methods. Topics include least squares, asymptotic theory,
hypothesis testing, instrumental variables, difference-in-differences,
regression discontinuity, treatment effects, panel data, maximum likelihood, and
discrete choice models.
}

\end{frame}


\begin{frame}{Prerequisites for Studying Econometrics}
Ideally, you should have experience with
\begin{itemize}
	\item Calculus
	\item Basic Probability and Statistics 
	\item Linear Algebra 
	\item Data analysis software such as R, Python/Numpy, Matlab, Julia, Stata, or similar
		(Excel doesn't count) 
\end{itemize}
\end{frame}
% 4 people missing LA, 4 missing programming training, 1 missing prob stats, only one person missing two things 
% for those without linear algebra: study our brief review, check out the khan academy thing soon if you haven't already 
% in particular the first two components: vectors and spaces, and matrix transformations
% you won't need every detail for this course, so be ready, when you come across something in the book you don't understand,
% go look it up and have some patience to make sure you fully understand what's going on. Same for lectures, if there was a 
% point involving matrices you didn't understand, be sure to look it up after and figure out what's going on. And of course 
% you can come to me if you have trouble figuring something out.
% Don't come to me saying "teach me linear algebra", I'm expecting more independence from you than that, but you are very welcome
% meet with me if you get stuck on something
% 
% For those with little or no programming experience, the best way to learn is to do it. If you're starting from scratch,
% or from very little, I suggest using R. Check out an R tutorial, and write your first program this weekend if you can. 


\begin{frame}{Assignments and Grading}
\begin{itemize}
	\item 4 Problem sets: 10\% each
		\begin{itemize}
		\item You may use any software you like. When giving examples, I will use R.
		\item To compose, consider using software like LaTeX, R Markdown, and Jupyter.
	\end{itemize}

	\smallskip
	\item Two in-class quizzes: 10\% each


	\smallskip
	\item Group project: 40\%
	\begin{itemize}
		\item A research project on any topic (subject to my approval) using econometric analysis.
		\item I suggest choosing a published paper to replicate and finding a way to extend, test, or improve on it.  
		Ideally, choose something on a topic that interests you. I will also share a list of suggestions. 
		\item 1-3 students per group
		\item Proposal due in middle of semester. 
		\item Presentations in final class session
		\item Paper due at end of semester
	\end{itemize}

\end{itemize}
\end{frame}







\begin{frame}{What is Econometrics?}
\begin{itemize}
	\item {\bf Experiments and Research Design:}
	\begin{itemize}
			\item In natural sciences, randomized controlled trials are considered the gold standard.
			\medskip
			\item In the social sciences, it's often hard to run experiments: macroeconomic policy, 
			mergers and antitrust policy.
			\medskip
			\item This is perhaps the main reason we have econometrics: in the absence of controlled 
			experiments, we need to figure out how to learn what we want to learn from naturally 
			occurring data.
	\end{itemize}
\end{itemize}
\end{frame}	
	
\begin{frame}{What is Econometrics?}
\begin{itemize}
	\item {\bf Econometric Questions:}
	\begin{itemize}
			\item Often about \emph{causality}
			\begin{itemize}
				\item About individuals: \emph{What is the effect of education on wages?}
				\item About markets (micro):   \emph{How does the price of the iPad affect the number of units that will be sold?}
				\item About markets (macro): \emph{How does raising the minimum wage affect employment?}
			\end{itemize}
			
			\bigskip
			\item Some studies are primarily descriptive 
			\begin{itemize}
				\item \emph{How many home runs will Aaron Judge hit in 2025?}
				\item \emph{How has the distribution of real income in the US changed since 1990?}
			\end{itemize}

	\end{itemize}				

\end{itemize}
\end{frame}

\begin{frame}{What is Econometrics?}
\begin{itemize}	

	\item{\bf Challenges:}
		\begin{itemize}
				\item Endogeneity: general term for an observed variable's being correlated with things we can't observe.
				Related: omitted variables. 
				\medskip
				\item Selection: economic agents (people, firms, etc.) are purposeful and know more than we do about their personal situations!
				\medskip
				\item Simultaneity: is the relationship between price and quantity increasing (supply) or decreasing (demand)? 
				 What is driving the changes? 
				The world is complicated, and it's often too simplistic to say \emph{`'this is the effect of X on Y''}. 
				
			\medskip
			\item External validity:
			\begin{itemize}
				\item What will happen if we raise the minimum wage to levels not before seen? 
				\item Is the price variation in the data short-run or long-run? Do consumers/firms respond differently to 
				the two types of variation?
				\item Related: structural econometrics, theory building. 
			\end{itemize}
		\end{itemize}	
\end{itemize}
\end{frame}




\end{document}
