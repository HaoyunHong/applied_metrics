\input{../preamble.tex}

\setbeamersize{text margin left=10pt,text margin right=10pt}


\begin{document}


\title[L5-Wage Equation Example]{ Econometrics I}
\subtitle{Lecture 5: Extended Example: The Wage Equation}
\author{Chris Conlon}
\date{Fall 2025}
\maketitle






\begin{frame}{Mincerian Regression}
\begin{itemize}
	\item Recall the Mincerian regression (wage equation):\[
	\ln wage_i = \beta_0 + \beta_{ed} Education_i + \beta_{exp} Experience_i + \beta_{Fem} Female_i + \dots + \varepsilon_i
	\]

	\smallskip
	\item Let's revisit estimating this with the Cornwell and Rupert (NLSY) data. 
\end{itemize}
\end{frame}




\begin{frame}{Process the data}
\begin{center}
	\includegraphics [width=.9\textwidth]{code_0}\\
\end{center}
\end{frame}



\begin{frame}{Baseline Results}
\begin{columns}
\begin{column}{0.6\textwidth}
\begin{figure}
\flushleft
	\includegraphics [width=\textwidth]	{code_1}
\end{figure}
Note on interpreting effects with log dependent variable:

Intrepreting coefficients for $\log(y_i)\approx 1+\beta$:
\begin{itemize}
	\item $\exp \left( -.3892\right) = .6826$
		\item $\exp \left( .05654\right) = 1.057$
\end{itemize}

\end{column}
\begin{column}{0.4\textwidth}
\begin{figure}
\flushleft
	\includegraphics [width=\textwidth]	{table_1}
\end{figure}
\end{column}
\end{columns}	

\end{frame}


\begin{frame}[t]
\frametitle{Heuristic Policies}

\begin{columns}

\column{0.5\textwidth}
\begin{itemize}
\item<1-> These are methods aiming to give a good but not necessarily optimal solution to a problem.
\item<2-> There exist a number of such policies for bandit problems.
\alt<1-9>{
\item<3-9> Greedy policy:
\begin{itemize}
\normalsize
\item<4-9> choose arm with greatest expected reward
\item<5-9> ignores variability in prior distribution
\item<6-9> quite good for Bernoulli bandits, but less effective for normal bandits
\end{itemize}
}{\only<10-14>{\item<10-> Next policy:
\begin{itemize}
\normalsize
\item<11-14> comment 1
\item<12-14> comment 2
\item<13-14> comment 3\par
\rule{0pt}{2.7cm}
\end{itemize}}
\only<15-18>{\item<15-> Another policy:
\begin{itemize}
\normalsize
\item<16-18> Another comment 1
\item<17-18> Another comment 2
\item<18-18> Another comment 3\par
\rule{0pt}{2.7cm}
\end{itemize}}
}
\end{itemize}

\column{0.3\textwidth}
\vspace{-25pt}
\uncover<7->{\begin{figure}
\begin{center}
\includegraphics[height = 2.7cm, trim=-1cm 0cm 0cm 0cm,clip=true,width=3cm]{example-image-a}
\caption*{$(\alpha,\beta) = (1,1)$}
\end{center}
\end{figure}}
\vspace{-25pt}
\uncover<8->{
\begin{figure}
\begin{center}
\includegraphics[height = 2.7cm, trim=-1cm 0cm 0cm 0cm,clip=true,width=3cm]{example-image-a}
\caption*{$(\alpha,\beta) = (6,5)$}
\end{center}
\end{figure}}

\column{0.2\textwidth} 
\only<9>{$\Rightarrow$ play this arm} 
\only<13-14>{$\Rightarrow$ play this arm with probability $\varepsilon$\vspace{80pt}} 
\only<14>{$\Rightarrow$ play this arm with probability $1-\varepsilon$} 
\end{columns} 
\onslide<10>{\null}
\end{frame}

\end{document}